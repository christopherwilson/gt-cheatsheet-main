\wde{1.3.1 Subgroup} If $H \subseteq G$ is nonempty then $H$ is a \textit{subgroup} provided that: i) $hk \in H\ \forall\ h, k \in H$; ii) $h^{-1} \in H$ for each $h \in H$. $H$ is subgroup $\Leftrightarrow$ $H \le G$.
\wde{1.3.6 Cosets} Let $H \le G$ and let $g \in G$. The \textit{left coset} of $H$ by $g$ is the set $gH := \{gh\ :\ h \in H\}$. Similarly, the \textit{left coset} of $H$ by $g$ is the set $Hg := \{hg\ :\ h \in H\}$. The set of left cosets is denoted $G/H$, the set of right cosets is $H\backslash G$. The index of $H$ in $G$ is $[G : H] = |G/H|$
\wde{1.3.7 Normal} A subgroup $H \le G$ is \textit{normal} if $gH = Hg\ \forall\ g \in G$. In this case we write $H \norm G$.
\wt{1.3.8 Lagrange's} Let $H \le G$, where $G$ is finite. Then $|G| = [G:H] \cdot |H|$
\wt{1.3.9 Cauchy's} If $G$ is a finite group, and $p$ is a prime that divides $|G|$, then $|G|$ has a subgroup of order $p$.
\wde{1.3.10 Order of $g$} Let $g \in G$. The \textit{order} of $g$ is the least positive integer s.t. $g^n=e$, or $\infty$ if no such $n$ exists. We write the order of $g$ as $o(g)$. $o(g) = |\cyc{g}|$.
\wc{1.3.11} If $|G|$ is prime, then $G$ is cyclic.