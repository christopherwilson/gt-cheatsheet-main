\wde{5.2.1} Let $R$ be a ring. An $R$-module is an abelian group $(M,+)$ together with a mapping $R \times M \to M$, $(r, a)\mapsto ra$ that is \textit{distributive}, \textit{associative}, and $1a =a $ for all $a \in M$ (\textit{unital}).
\wde{5.2.4 Free $R$-Module} Let $R$ be a ring and let $n \in \mathbb{N}$. The \emph{free $R$-module of rank $n$} is the $n$-fold cartesian product $R^n$. It is given a module structure by $r(a_1, a_2, \ldots, a_k) = (ra_1, ra_2, \ldots, ra_k)$.
\wt{5.2.5 Fundamental Theorem of Finitely Generated Abelian Groups} Let $A$ be a finitely generated abelian group. Then $A \cong \mathbb{Z}/r_1\mathbb{Z} \times \mathbb{Z}/r_2\mathbb{Z} \times \cdots \times \mathbb{Z}/r_k\mathbb{Z} \times \mathbb{Z}^{\ell}$, for some $k,\ell \in \mathbb{N}$ and $r_1, \ldots, r_k$ nonzero elements of $\mathbb{Z}$ with $r_1|r_2|\ldots|r_k$.
\wpf{} $A$ is finitely generated so $A \cong \mathbb{Z}^s/K$ for some $s \in \mathbb{N}$ and $K \subseteq \mathbb{Z}^s$. $K$ is finitely generated by $x_1, \ldots, x_r \in \mathbb{Z}^s$. Write $x_i = \sum^s_{j=1} a_{ij}e_j$, where $e_j$ is the $j$-th standard basis vector. Let $M = (a_{ij} \in M_{r \times s}(\mathbb{Z})$. By prop. 5.2.7 we can apply invertible row and column operations to $M$ without changing the isomorphism class of $A$. Goal is to use invertible row and column operations to change $M$ to a diagonal matrix. 
\textbf{Step 0:} if $M = \mathbf{0}$, we are done, $A \cong \mathbb{Z}^s$.
\textbf{Step 1:} if $M \ne \mathbf{0}$, by swapping rows and columns we can make $a_{11} \ne 0$.
\textbf{Step 2:} We \textit{clean} column 1, make it $(b, 0, \ldots, 0)^T$, where $b = \gcd(a_{11}, a_{21}, \ldots, a_{r1})$, begin by
\textbf{Step 2a:} Make $a_{21} = 0$ by replacing $a_{11}$ by $b = \gcd(a_{11}, a_{21}) = (b)$, which means there are $x, y, p, q \in \mathbb{Z}$ so that: $a_{11}x + a_{21}y = b$ by the Euclidean Algorithm, $a_{11} = pb$ since $b|a_{11}$, $a_{21} = qb$ since $b|a_{21}$. Thus $pbx = qby = b$, and $px + qy = 1$, which means matrix $D = \begin{pmatrix}
    x & y \\
    -q & p
\end{pmatrix}$ is invertible, with inverse given by the adjoint matrix $D^{-1} = \begin{pmatrix}
    p & -y \\
    q & x
\end{pmatrix}$. Further, $D(a_{11}, a_{21})^T = (b, -qpb +pqb = 0)^T.$ Therefore transforming $M \leadsto D'M$ does not change $A$. Repeat 2a on rows 1 and 3, then 1 and 4 and so on. 
\textbf{Step 3:} Use right multiplication by invertible matrices to clean row 1 until it is $(c, 0, \ldots, 0)$, with $c$ equal to the $\gcd$ of the top row entries. This might change $b$, but $c|b$.
\textbf{Step 4:} Cleaning top row may mess up left column, so we may need to repeat step 2, then repeat step 3, and so on until all columns are clean. Each time we change $a_11$ we replace it by a divisor, so we get the sequence $a_{11}, b, c, \ldots$, where each integer divides the previous one. 5.3.1 tells us after a finite number of steps both the row and column will be clean.
\textbf{Step 5:} If the new $a_{11}$ does not divide all $a_{ij}$, add column $j$ to column 1 and go back to step 2. Repeat until we have an $a_{11}$ that divides all entries. Repeat again but with the submatrix with $a_{22}$ in the top left corner, and so on until $M$ is a diagonal matrix, where $r_i = a_{ii}$, and there are $k$ elements in the first entries of the main diagonal, and 0 everywhere else. We have that $A \cong \mathbb{Z}^s/K'$, where $K' = \sum^k_{i=1} \mathbb{Z}(r_i e_i)$. This means $A' \cong \mathbb{Z}/(r_1) \times \mathbb{Z}/(r_2) \times \cdots \times \mathbb{Z}/(r_k) \oplus \mathbb{Z}^{s-k}$.
\wl{5.2.6} Let $\alpha$ be a $\mathbb{Z}$-module automorphism of $\mathbb{Z}^s$. Then $\mathbb{Z}^s/K \cong \mathbb{Z}/\alpha(K)$.
\wpr{5.2.7} Suppose that $M$ is the $r \times s$ matrix corresponding to $K = \sum^{r}_{i=1} \mathbb{Z}x_i \subseteq \mathbb{Z}^s$. If we change $M \rightsquigarrow M'$ via invertible row and column operations, then $M'$ corresponds to a submodule $K'$ of $\mathbb{Z}^s$ so that $\mathbb{Z}^s/K \cong \mathbb{Z}^s/K'$.
