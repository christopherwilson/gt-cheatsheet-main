\wt{6.3.1} The alternating group $A_5$ is simple.
\wt{6.3.3} Let $n \ge 5$. Then $A_n$ is simple.
\wpf{} Induct on $n$, base case $n = 5$ given by Thm. 6.3.1. Suppose $n \ge 6$ and $H \norm A_n$. Will show $H = A_n$ or $H = \{()\}$. 
For $1 \le i \le n$, let $B_i = \{\sigma \in A_n : \sigma(i) = i\}.$ Then $B_i \cong A_{n-1}$. Since $H \cap B_i \norm B_i$ and $B_i \cong A_{n-1}$, we have $H \cap B_i$ is either $\{()\}$ or $B_i$.
\textbf{Case 1:} $H \cap B_i = B_i$ for some $i$, then $B_i \subseteq H$, and in particular $H$ contains a 3-cycle. By lemma 6.3.4 all 3-cycles are conjugate in $A_n$. As $H \norm A_n$ we have that $H$ contains all 3-cycles in $A_n$. By Lemma 6.3.5, $A_n$ is generated by 3-cycles, so $H = A_n$
\textbf{Case 2:} $H \cap B_i = \{()\}$ for $1 \le i \le n$, meaning if $\sigma \in H$ with $\sigma \ne ()$ then $\sigma$ does not fix any $i$. Therefore, by Lem. 6.3.6, $|H| \le n$. Suppose there is $\sigma \in H$ with $\sigma \ne ()$, by Lem. 6.3.7 $|\Cl_{A_n}(\sigma)| \ge n$. Since $\Cl_{A_n}(\sigma) \cup \{()\} \subseteq H$, we have $|H| \ge n+1$, a contradiction so $H = \{()\}$. $\qed$
\wl{6.3.4} If $n \ge 5$ and $\sigma, \sigma'$ are 3-cycles in $A_n$, then $\sigma$ and $\sigma'$ are conjugate in $A_n$: that is, there exists $\tau \in A_n$ with $\tau \sigma \tau^{-1} = \sigma'$.
\wl{6.3.5} If $n \ge 3$, then $A_n$ is generated by 3-cycles.
\wl{6.3.6} If $H \le S_n$ and $H$ has the property that any $\sigma \in H$ with $\sigma \ne ()$ is fixed-point-free, then $|H| \le n$.
\wl{6.3.7} If $n \ge 6$ and $\sigma \in A_n$ with $\sigma \ne ()$, then $|\Cl_{A_n}(\sigma)| \ge n$.