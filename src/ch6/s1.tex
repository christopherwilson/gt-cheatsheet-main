\wl{6.1.1} Every permutation can be written as a product of disjoint cycles, and the product is unique up to re-ordering the factors.
\wde{6.1.X Transpositions} A 2-cycle is often called a \textit{transposition}, and a 2-cycle of the form $(i\ i+1)$ is an \textit{adjacent transposition}. 
\wl{6.1.2} Every permutation can be written as a product of transpositions. Thus, $S_n$ is generated by transpositions.
\wde{6.1.3 Cycle Type} Suppose that $\sigma = c_1 c_2 \ldots c_k$ is a product of $k$ disjoint cycles of lengths $l_1, l_2, \ldots, l_k$ with $l_1 \ge l_2 \ge \cdots \ge l_k$. Then the $k$-tuples $(l_1, l_2, \ldots, l_k)$ is called the \emph{cycle type} of $\sigma$.
\wl{6.1.7} Let $\sigma = (a_1\ a_2\ \cdots\ a_k) \in S_n$, and let $\tau \in S_n$. Then 
$\tau \sigma \tau^{-1} = (\tau(a_1)\ \tau(a_2)\ \cdots\ \tau(a_k)).$
\wt{6.1.8} Two permutations in $S_n$ are conjugate if and only if they have the same cycle type.