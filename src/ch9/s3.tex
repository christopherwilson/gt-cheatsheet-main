\wde{9.3.1 Normal Extension} $L/K$ is called a \emph{normal extension} if it is generated by all the roots of a set of irreducible polynomials.
\wl{9.3.2} If $L/K$ is a normal extension, and $x \in L$ is a root of the irreducible polynomial $f(X) \in K[X]$, then $f(X)$ splits over $L$. In other words, if one considers $L$ as a subfield of $\overline{K}$, then all the roots of $f(X)$ are in $L$. As a result, $\Aut(L/K)$ acts transitively on the roots of $f$.
\wde{9.3.3 Galois Extensions/Group} We refer to normal extensions also as \emph{Galois extensions}, and the automorphism group also as the \emph{Galois group}.
\wt{9.3.4} If $L/K$ is a finite Galois extension, then $H \mapsto L^H$ and $K \subset E \subset L \mapsto \Aut(L/E) \le \Aut(L/K)$ gives a correspondence between the subgroups and the intermediate fields between $K$ and $L$.