\wde{4.1.1 Sylow $p$-subgroup} Let $G$ be a finite group and let $p$ be a prime. A subgroup $H$ of $G$ is a \textit{$p$-subgroup of $G$} if it is a $p$-group, that is it has order $p^n$ for some $n$, and it is a \textit{Sylow $p$-subgroup of $G$} if its order is the highest power of $p$ that divides the order of $G$. We say that $H$ is a \textit{Sylow subgroup of $G$} if it is a Sylow $p$-subgroup for some prime $p$.
\wt{4.1.2 Sylow I} Let $|G| = n$ and suppose that $p$ is a prime that divides $n$. Write $n = p^m r$ with $p$ not dividing $r$. Then there exists at least one Sylow $p$-subgroup of order $p^m$.
\wpf{} Let $X$ be the set of all \textbf{subsets} $A$ of $G$ with $|A| = p^m$. If $g \in G$ and $A \in X$ then $gA = \{ga : a \in A\}$ has $p^m$ elements since the map $A \to gA$, $a \mapsto ga$ is bijective. So $gA \in X.$ Thus, $G$ acts on $X$ by left translation: $g \cdot A = gA$. Suppose $A \in X$ has an orbit of this action with size not divisible by $p$. Since $|G \cdot A| = [G:\Stab_G(A)]$ by Theorem 4.2.6 this means that $p$ does not divide $[G : \Stab_G(A)]$, however $p^m r = |G| = [G:\Stab_G(A)]|\Stab_G(A)|$, so $p^m$ divides $|\Stab_G(A)|$.
Choose any $a \in A$, then $\Stab_G(A)a \subseteq A$, since $\Stab_G(A) \cdot A = A$. Thus, $|\Stab_G(A)| = |\Stab_G(A)a| \le |A| = p^m$, hence $|\Stab_G(A)| = p^m$, so $\Stab_G(A)$ is a $p$-subgroup. To prove existence of an orbit with size not divisible by $p$, we have $|X| = \binom{p^mr}{p^m} = \frac{p^m r(p^m r -1)\ldots (p^mr -(p^m-1))}{p^m(p^m -1)\ldots (p^m -(p^m-1))}$. Let $1 \le s \le p^m -1$, and consider $p^m r - s$ and $p^m -s$. The highest power of $p$ dividing both is the highest power of $p$ dividing $s$, and the highest power of $p$ dividing $p^mr$ is $p^m$. Therefore, the numerator and denominator of $|X|$ is divisible by the same power of $p$; which cancel so $p$ does not divide $|X|$. Since $X$ decomposes as a disjoint union of orbits, there must be one orbit $G \cdot A$ with $p$ not dividing $|G \cdot A|$. 
%
\wt{4.1.3 Sylow II} Let $|G| = n$ and suppose that $p$ is a prime that divides $n$. Write $n = p^m r$ with $p$ not dividing $r$. Suppose that $P$ is a Sylow $p$-subgroup and that $H \le G$ is any $p$-subgroup of $G$. Then there exists $x \in G$ with $H \subseteq xPx^{-1}$. In particular, any two Sylow $p$-subgroups of $G$ are conjugate in $G$.
\wpf{} The $p$-group $H$ acts by left translation on $G/P$: that is $h \cdot (gP) := hgP$. By Lemma 4.3.1 the number of fixed points of this action is congruent modulo $p$ to $|G/P| = [G : P] = r$. As $P$ is a Sylow $p$-subgroup of $G$, $p$ does not divide $[G:P]$ so there must be at least one fixed point $xP$. $hxP = xP$ $\forall h \in  H$, and so $x^{-1}hx \in P$ $\forall h \in H$, thus $x^{-1}Hx \subseteq P$, thus $H \subseteq xPx^{-1}$. Suppose $H$ is a Sylow $p$-subgroup, then $|H| = |P| = |xPx^{-1}|$. Since $H \subseteq xPx^{-1}$ this forces $H = xPx^{-1}$, so $H$ is conjugate to $P$.
% 
\wt{4.1.4 Sylow III} Let $|G| = n$ and suppose that $p$ is a prime that divides $n$. Write $n = p^m r$ with $p$ not dividing $r$. Let $n_p$ be the number of distinct Sylow $p$-subgroups of $G$. Then $n_p | r$ and $n_p \equiv 1$ modulo $p$. 
\wpf{} $n_p = [G: N_G(P)] = |G|/|N_G(P)|$ by the previous lemma. Since $r = \frac{|G|}{|P|} = \frac{|G|}{|N_G(P)|}\frac{|N_G(P)|}{|P|}$, $n_p | r$. Let $X$ be the set of all Sylow $p$-subgroups of $G$. Choose $p \in X$. Let the $p$-group $P$ act on $X$ by conjugation, by lemma 4.3.1 the number of fixed points for this action is congruent to $|X| = n_p \mod p$. 
$P$ is fixed under this action since $pPp^{-1} = P$ $\forall p \in P$, claim that $P$ is the only fixed point.
Suppose $Q$ is a fixed point, then $pQp^{-1} = Q\ \forall p \in P$; so $P \subseteq N_G(Q)$. Therefore $P$ and $!$ are Sylow $p$-subgroups of $N_G(Q)$. However, $Q \norm N_G(Q)$; and so $Q$ is the only Sylow $p$-subgroup of $N_G(Q)$ by Sylow II. Hence, $P = Q$. 
By Lemma 4.3.1, $n_p = |X| \equiv 1 \mod p$.
%
\wde{4.1.8 Simple}  A group $G$ is \textit{simple} if $G$ has no nontrivial normal subgroups.
\wl{4.1.9} If a group $G$ has a unique Sylow $p$-subgroup $P$, then $P \norm G$.